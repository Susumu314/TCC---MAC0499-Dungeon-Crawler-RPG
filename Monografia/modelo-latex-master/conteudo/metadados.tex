%!TeX root=../tese.tex
%("dica" para o editor de texto: este arquivo é parte de um documento maior)
% para saber mais: https://tex.stackexchange.com/q/78101/183146

% Insira aqui os metadados do seu trabalho. Para isso, copie,
% com as alterações necessárias, o conteúdo do arquivo
% conteudo-exemplo/metadados.tex
% Estes comandos definem o título e autoria do trabalho e devem sempre ser
% definidos, pois além de serem utilizados para criar a capa, também são
% armazenados nos metadados do PDF.
\title{
    % Obrigatório nas duas línguas
    titlept={Jogo digital focado na exploração de labirintos e captura de criaturas},
    titleen={Digital game focused on dungeon crawling and criature collecting},
}

\author{Hugo Susumu Mota Asaga}

% Para TCCs, este comando define o supervisor
\orientador{Prof. Ricardo Nakamura}

% A página de rosto da versão para depósito (ou seja, a versão final
% antes da defesa) deve ser diferente da página de rosto da versão
% definitiva (ou seja, a versão final após a incorporação das sugestões
% da banca).
\defesa{
  nivel=tcc, % mestrado, doutorado ou tcc
  % É a versão para defesa ou a versão definitiva?
  %definitiva,
  % É qualificação?
  %quali,
  programa={Ciência da Computação},
  local={São Paulo},
  data=2022-02-06, % YYYY-MM-DD
  % A licença do seu trabalho. Use CC-BY, CC-BY-NC, CC-BY-ND, CC-BY-SA,
  % CC-BY-NC-SA ou CC-BY-NC-ND para escolher a licença Creative Commons
  % correspondente (o sistema insere automaticamente o texto da licença).
  % Se quiser estabelecer regras diferentes para o uso de seu trabalho,
  % converse com seu orientador e coloque o texto da licença aqui, mas
  % observe que apenas TCCs sob alguma licença Creative Commons serão
  % acrescentados ao BDTA.
  direitos={CC-BY}, % Creative Commons Attribution 4.0 International License
  %direitos={Autorizo a reprodução e divulgação total ou parcial
  %          deste trabalho, por qualquer meio convencional ou
  %          eletrônico, para fins de estudo e pesquisa, desde que
  %          citada a fonte.},
  % Isto deve ser preparado em conjunto com o bibliotecário
  %fichacatalografica={nome do autor, título, etc.},
}

% As palavras-chave são obrigatórias, em português e
% em inglês. Acrescente quantas forem necessárias.
\palavrachave{Jogo Digital}
\palavrachave{Videogame}
\palavrachave{RPG}
\palavrachave{Pokémon}
\palavrachave{Dungeon Crawler}
\palavrachave{Game Design}

\keyword{Digital Game}
\keyword{Videogame}
\keyword{RPG}
\keyword{Pokémon}
\keyword{Dungeon Crawler}
\keyword{Game Design}

% O resumo é obrigatório, em português e inglês.
\resumo{
 O objetivo deste trabalho foi desenvolver o protótipo de um jogo digital, do gênero \emph{Role-Playing Game} (RPG), \emph{Dungeon Crawler}, que busca agradar os jogadores que gostam de jogos de captura e criação de criaturas, e jogos RPG que seguem convenções estabelecidas nas décadas de 1980-1990. A partir da análise de jogos bem avaliados pela crítica se criou o design base do jogo, o qual foi implementado durante a fase de desenvolvimento e refinado por meio de \emph{feedbacks} recebidos durante o primeiro teste com público. Futuramente é possível levar adiante o desenvolvimento do jogo baseado nos sistemas já criados para adicionar mais conteúdo ao jogo.
}

\abstract{
The goal of this work was to develop a prototype of a digital game of the genre Role-Playing Game, Dungeon Crawler, that seeks to target the players that like games where you capture and train creatures and RPG games that follow the conventions set in the 80s and 90s. By analyzing well-received games, the base design of the game was created, which was implemented during the development phase and was polished using the feedback gathered during the first player test. Subsequently, it is possible to continue the development of the game based on the systems already created to add more content to the game.
}
